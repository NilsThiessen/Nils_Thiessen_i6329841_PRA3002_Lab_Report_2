\newpage
\section{Discussion}
\label{Discussion}

During the experiment, the gas inside an apparatus consisting of a metal can connected to a container with a movable piston was checked for its ideality. The goal was met by varying the temperature $\mathit{T}$ at constant volume $\mathit{V}$ and observing the changes in pressure $\mathit{P}$. Afterwards, the system underwent an Ericsson cycle, where the efficiency of the heat engine was calculated and compared to the ideal one found during a Carnot cycle. Section \ref{Theory} contains a few predictions towards the experiment's outcome.

\subsection{Ideal Gas Law Experiment: Discussion}
\label{Ideal Gas Discussion}

When inserting the can into the heat baths, one could observe a sudden spike in pressure, followed by a steady decline towards room temperature levels. One can attribute the decline to the leakage of molecules. Especially more pronounced at higher temperatures and therefore pressures, the transition balances the pressures inside and outside.

From the collected data, it was possible to verify the positive linear correlation between pressure and temperature. The data points, when considering their uncertainties, all lie on a linear function fitted to it. Additionally, approximated to two decimal digits, the correlation coefficient was found to be $\mathit{R=1.00}$, visualizing the approximate ideal behaviour of the air contained in the system, in line with the prediction.

\subsection{Thermodynamic Cycle: Discussion}
\label{Thermodynamic Cycle Discussion}

The Ericsson cycle was based on two heat reservoirs at $\mathit{T_c} ) 284.29 K$ and $\mathit{T_h} ) 324.76 K$, with a net work of $\mathit{W=0.0618J}$. Exploiting constant quantities during specific processes allowed the determination the heat $\mathit{Q_h}$ absorbed in steps II and III. Summing them up yields $\mathit{Q_h = (12.6059 \pm 0.8219) J}$.

The heat engine's actual efficiency was thus found to be $\mathit{\nu = 0.00485 \pm 0.00032}$, with the ideal Carnot efficiency relative to the temperatures $\mathit{T_c}$ and $\mathit{T_h}$ being $\mathit{\eta 
= 0.12186 \pm 0.0028}$. The heat engine was therefore only able to achieve $\mathit{3.98 \%}$ of its maximum efficiency, which changes little when taking the uncertainties into account. 

Although somewhat expected, the heat engine appears to be highly impotent in producing apt work for the heat it absorbs. The simple materials could be an explanation, but specific precautions had been taken to prevent leaks. One fault can be found in the analysis. Paths I and III where described as being isothermal. Frankly, such kind of processes normally occur at slow speeds, so that the system does not experience changes in internal energy, keeping the temperature constant. However, altering the calculations for process III can do little to decrease the heat absorbed by the system, as process II contributes most to $\mathit{Q_{h}}$. 

\subsection{Changes to Experiment: Reducing the Uncertainties}
\label{Uncertainty Reduction}

Although the sample size was rather small, the results from the ideal gas experiment appear promising, in that they verify Gay-Lussac's law for the case of the air trapped in the metal can. That being said, the uncertainties on temperature have to be decreased for more precise measurements. For the coldest bath, the temperature is only known with 5 \% uncertainty. While the temperature probes themselves seem to work accurately, the Quad Temperature Sensor, with an accuracy of $\mathit{\pm 0.5K}$, could have considerable impact on the results. More precise temperature measurements therefore require a replacement.

For the Ericsson Cycle Experiment, numerous improvements can be made. For one, the determining the initial position $\mathit{y_0}$ of the piston inside the glass cylinder was done rather carelessly, with the reading guesstimated from the scale printed on the cylinder and the initial position changing throughout each cycle, with the lack of proper methods to evaluate these changes. Instead, one could have calibrated the Motion Sensor by completely compressing the piston and setting that position as 0. The reading would be accurate, and the subsequent cycles could be appropriately compared.

The previous suggestion already substantially decreases the uncertainty of measurement. Nonetheless, the error in $\mathit{Q_{III}}$ would still be considered significant. The degree thereof mostly originates from the logarithmic function. $\mathit{V_3}$ and $\mathit{V_4}$ deviate only little from one another, so small changes can have large impact. Between equilibrium points 3 and 4, the mass is removed from the system, meaning that implementing heavier masses imply bigger volume changes. However, the leaks would worsen as a result. Generally increasing the scale of the experiment would not alter the ratio $\mathit{\frac{V_4}{V_3}}$, the suggestion would thus be ineffective. As such, unless the problem with leakage in the system is fixed, more precise volumes remain the most constructive improvements. 

\subsection{Further experiments}
\label{Further experiments}

Generally, the ideal gas law experiment can be deemed successful, as the data fit almost perfectly to a linear function. One can thus apply other laws on the apparatus, such as Boyle's law or Charles' law. Under the conditions of the experiment conducted in this paper, the results should similarly be in agreement with theory. One could additionally implement higher temperatures, to observe when air stops acting ideal. The same idea could be implemented more easily by fixing the piston at smaller volumes. In such a scenario, pressure would increase more quickly under temperature changes. If the current apparatus is apt for such conditions remains to be seen, with its leaks being especially apparent at high pressure. 

Furthermore, as of now, only the thermal efficiency of the cycle has been discussed. However, one can also consider the actual usable work compared to the net work of the process. This would be seen in the displacement of the piston-mass complex, where a force equal to that of the mass' weight force would be acted over a certain distance. An interesting to observe how much of the net work is actually converted into the displacement work, meaning that one could find the mechanical efficiency. 

