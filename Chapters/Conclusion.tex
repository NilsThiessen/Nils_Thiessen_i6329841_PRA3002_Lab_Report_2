\newpage
\section{Conclusion}
\label{Conclusion}

The first experiment posed the question if the air contained in the apparatus can be considered ideal. By showing that pressure and temperature had a linear correlation with correlation coefficient $\mathit{R=1.00}$ when the other variables where held constant, Gay-Lussac's law was validated and the apparatus could be safely implemented for a thermodynamic cycle. 

The second experiment aimed at replicating Ericsson's cycle. From the data, the net work per cycle $\mathit{W}$ and the heat $\mathit{Q_h}$ absorbed were found, meaning that one could calculate the efficiency. The heat engine's actual efficiency was determined as $\mathit{\nu = 0.00485 \pm 0.00032}$, while the theoretical efficiency of a Carnot cycle would have been $\mathit{\eta = 0.12186 \pm 0.00028}$. The heat engine therefore worked at only $\mathit{3.98 \%}$ of its maximal thermal efficiency. 

Suggestions have been made to improve on the accuracy of the measurement. These contain finding a precise reference point for the volume inside the glass cylinder of the piston by calibrating the Rotary Motion Sensor, or observing more points for the ideal gas law experiment. Additionally, further experiments have been brought forth, namely that of calculating the mechanical efficiency or verifying the ideality of the system's air by checking other laws. 